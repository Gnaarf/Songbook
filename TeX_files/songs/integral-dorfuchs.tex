\songtitle{Integral}{DorFuchs}{2013}

\guitarchord{C}
\guitarchord{Am}
\guitarchord{Dm}
\guitarchord{G}
\guitarchord{Em}
\guitarchord{F}
\guitarchord{Fmaj7}
\guitarchord{D}
\guitarchord{Bm}
\guitarchord{A}

\begin{guitar}
	\songsection{Strophe 1}
	Das unbestimmte Inte[C]gral von f(x) d[Am]x ist weiter nix
	Als die [Dm]Menge aller Stammfunktionen, deren [G]erste Ableitung f(x) ergibt und
	[C] Das bestimmte Inte[Am]gral ist eine reelle [Dm]Zahl
	Die so groß ist, wie die [G]Fläche unter dem Graphen
	Doch, wenn der [Dm]Graph unterhalb der [Em]x-Achse ist, liegt die [F]Fläche ziemlich [C]tief
	Und [Dm]dann ist das be[Am]stimmte Inte[Fmaj7]gral nega[G]tiv
	
	\songsection{Strophe 1 (nochmal)}
	Das unbestimmte Integral von f(x) dx ist weiter nix
	Als die Menge aller Stammfunktionen, deren erste Ableitung f(x) ergibt und
	Das bestimmte Integral ist eine reelle Zahl
	Die so groß ist, wie die Fläche unter dem Graphen
	Doch, wenn der Graph unterhalb der x-Achse ist, liegt die Fläche ziemlich tief
	Und [Dm]dann ist das be[Am]stimmte Inte[Fmaj7]gral nega[G]tiv [A]{}
	
	\songsection{Strophe 1 (bisschen anders)}
	Das unbestimmte Inte[D]gral von f(x) d[Bm]x ist weiter nix
	Als die [Em]Menge aller Stammfunktionen, deren [A]erste Ableitung f(x) ergibt und
	[D] Das bestimmte Inte[Bm]gral ist eine reelle [Em]Zahl
	Die so groß ist, wie die [A]Fläche, die unter dem Graphen l[D]iegt
	Nur, dass es, [C]wenn der Graph unterhalb der [G]x-Achse liegt
	Einen [D]negativen Wert ergibt.[C G D]{}
\end{guitar}