\songtitle{Auf dem Donnerbalken}{Volkslied}{$\sim$1980}
% ca. 1980(?) (quelle: https://www.volksliederarchiv.de/auf-dem-donnerbalken-sassen-zwei-gestalten-klopapier/)
% Herkunft: Das Wort ("Donnerbalken") ist in der Soldatensprache laut Küpper seit 1914 belegt. (quelle: https://de.wiktionary.org/wiki/Donnerbalken)

\guitarchord{C}
\guitarchord{G}

\begin{guitar}
	\songsection{Strophen}
	[C]Saßen zwei Gestalten auf dem Donnerbalken
	Und sie schrien nach [G]Klopapier - Klopapier!
	
	[G]Und da kam der Dritte, setzt sich in die Mitte
	Und sie schrien nach [C]Klopapier - Klopapier!
	
	
	Und da kam der Vierte, dass die Scheiße schmierte
	Und sie schrien nach Klopapier - Klopapier!
	
	Und da kam der Fünfte, der die Nase rümpfte
	Und sie schrien nach Klopapier - Klopapier!
	
	Und da kam der Sechste, als die Scheiße kleckste
	Und sie schrien nach Klopapier - Klopapier!
	
	Und da kam der Siebte, der den Anblick liebte
	Und sie schrien nach Klopapier - Klopapier!
	
	Und da kam der Achte, dass der Balken krachte
	Und sie schrien nach Klopapier - Klopapier!
	
	Und da kam der Neunte, als die Scheiße schäumte
	Und sie schrien nach Klopapier - Klopapier!
	
	[C]Und da kam der Zehnte, brachte das ersehnte
	[C]Klo-[G]pa-[C]pier!
	
	\optionalChord{[C]Und dann kam der Elfte, nahm sich gleich der Hälfte
	Und sie schrien nach [G]Klopapier - Klopapier!
	
	[G]Und dann kam der Zwölfte, nahm die and're Hölfte
	Und sie schrien nach [C]Klo-[G]pa-[C]pier!}
\end{guitar}