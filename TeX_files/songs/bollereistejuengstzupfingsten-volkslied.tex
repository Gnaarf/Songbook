\begin{spacing}{0.96}% to fit everything on one page.
\songtitle{Bolle reiste jüngst zu Pfingsten}{Volkslied}{$\sim$1900}

\guitarchord{G}
\guitarchord{C}
\guitarchord{D7}
\guitarchord{D}
\guitarchord{A7}

\begin{guitar}
	\songsection{Strophe 1}
	Bolle [G]reiste jüngst zu [C]Pfingsten, nach [D7]Pankow war sein [G]Ziel;
	Da ver[G]lor er seinen [C]Jüngsten janz [D7]plötzlich im Je[G]wühl;
	’Ne [D]volle halbe Stunde hat [A7]er nach ihm je[D]spürt.
	Aber [G]dennoch hat sich [C]Bolle janz [D7]köstlich amü[G]siert. \optionalChord{(x2)}
	
	\songsection{Strophe 2}
	In Pankow jab’s keen Essen, in Pankow jab’s keen Bier,
	War allet uffjefressen von fremden Leuten hier.
	Nich’ ma’ ’ne Butterstulle hat man ihm reserviert!
	Aber dennoch hat sich Bolle janz köstlich amüsiert. \optionalChord{(x2)}
	
	\songsection{Strophe 3}
	Auf der Schönholzer Heide, da jab’s ’ne Keilerei.
	Und Bolle, jar nich feige, war mittenmang dabei,
	Hat’s Messer rausjezogen und fünfe massakriert.
	Aber dennoch hat sich Bolle janz köstlich amüsiert. \optionalChord{(x2)}
	
	\songsection{Strophe 4}
	Es fing schon an zu tagen, als er sein Heim erblickt.
	Das Hemd war ohne Kragen, das Nasenbein zerknickt,
	Das rechte Auge fehlte, das linke marmoriert.
	Aber dennoch hat sich Bolle janz köstlich amüsiert. \optionalChord{(x2)}
	
	\songsection{Strophe 5}
	Als er nach Haus’ jekommen, da ging’s ihm aber schlecht,
	Da hat ihn seine Olle janz mörderisch verdrescht!
	’Ne volle halbe Stunde hat sie auf ihm poliert.
	Aber dennoch hat sich Bolle janz köstlich amüsiert. \optionalChord{(x2)}
	
	\songsection{Strophe 6}
	Und Bolle wollte sterben, er hat sich’s überlegt:
	Er hat sich uff die Schienen der Kleinbahn druffjelegt;
	Die Kleinbahn hat Verspätung und vierzehn Tage druff,
	Da fand man unsern Bolle als Dörrjemüse uff. \optionalChord{(x2)}
	
	\songsection{Strophe 7}
	Und Bolle wurd’ begraben, in einer alten Kist’.
	Der Pfarrer sagte ‚Amen‘ und warf ihn uff ’n Mist.
	Die Leute klatschten Beifall und gingen dann nach Haus.
	Und nun ist die Jeschichte von unserm Bolle aus!\vfill%
\end{guitar}
\end{spacing}