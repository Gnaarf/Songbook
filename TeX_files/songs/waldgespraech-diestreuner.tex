\songtitle{Waldgespräch}{Joseph von Eichendorff}{1815}
Melodie: Die Streuner

\guitarchord{Am}
\guitarchord{C}
\guitarchord{G}
\guitarchord{F}
\guitarchord{Em}

\begin{guitar}
	
	\songsection{Intro}
	{\footnotesize\begin{tabular}{l|l|l|l}
		Am C & G F & Am C & G \\
		Am C & G F & Am G & Am G Am
	\end{tabular}}
	
	\songsection{Strophe 1}
	Es [C]ist schon [G]spät, es [Am]wird schon [G]kalt.
	Was [F]reitest du einsam [G]durch den Wald?
	Der [C]Wald ist [G]lang, du [Am]bist al[Em]lein.
	Du [F]schöne Braut, ich führ dich [Am]heim!
	
	\songsection{Zwischenspiel}
	
	\songsection{Strophe 2}
	Groß ist der Männer Trug und List.
	Vor Schmerz mein Herz gebrochen ist.
	Wohl irrt das Waldhorn her und hin.
	Oh, flieh! Du weißt nicht, wer ich bin.
	
	\songsection{Instrumental}
	
	\songsection{Strophe 3}
	So reich geschmückt sind Ross und Weib,
	So wunderschön der junge Leib.
	Jetzt kenn' ich dich! Gott steh' mir bei!
	Du bist die Hexe Loreley.
	
	\songsection{Instrumental}
	
	\songsection{Strophe 4}
	Du kennst mich wohl vom hohen Stein,
	Schaut still mein Schloss tief in der Rhein.
	Es ist schon spät, es wird schon kalt.
	Kommst nimmer mehr aus diesem Wald!
	
	\songsection{Outro} (x2)
\end{guitar}