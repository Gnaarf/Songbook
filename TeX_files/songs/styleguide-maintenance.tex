\songtitle{Style Guide}{Maintenance}{2025}

\newcommand{\comment}[1]{\textcolor{teal}{\textsf{//#1}}}
\adjustboxDefault{%
% list chords in the order they appear in the song
\guitarchord{A}
\guitarchord{C}
% use another voicing (i.e. fingering) of a chord: 
\guitarchord{B-2}
% different voicing of the same chord: use optional parameter as suffix. Prefferably "[']".
\guitarchord[']{A-2}}
\strummingHeader{Verse}{\strummingTwoBeatsAndFourMovementsPerBeat{d.d.d.du}}
\strummingHeader{Chorus}{\strummingTwoBeatsAndFourMovementsPerBeat{b.d.b.dd}}

% strumming patterns/capo that don't fit in the chord's line: right allign
\hfill\capoEN{1}

% language of subsection headers: use either German (for German songs) or English (for all international songs)
\begin{guitar}
	\begin{highlightbar}
		% repetitions of songsection: greyed-out
		\songsection{Chorus (x2)}
		% capitalize first letter of every line
		[A]Lorem ipsum dolor sit amet, [C]consetetur sadipscing elitr, 
		[B]Sed diam nonumy eirmod tempor invidunt ut
		% chord after last word in line: add empty argument; otherwise line break won't work)
		[A'] Labore et dolore magna aliquyam erat, [C]{}
		% last line in "highlightbar" environment: add "%"; otherwise there will be an empty line
		[A]Sed diam voluptua.
	\end{highlightbar}
	
	\songsection{Verse 1}
	At vero eos et accusam et justo duo dolores et ea rebum. 
	Stet clita kasd gubergren, 
	No sea takimata sanctus est 
	% comments: greyed-out and in brackets
	Lorem ipsum dolor sit amet. \optionalChord{(pause for 2 beats)}
	
	% single line instrumental: directly after songsection
	\songsection{Instrumental} {\footnotesize\begin{tabular}{|l|l|l|l|}
			Em D & C & Em D & C 
	\end{tabular} \optionalChord{(x2)}}
	
	\begin{highlightbar}
		% repeated songsection: repeat the first line, then add "..."
		% repetitions of songsection: greyed-out
		\songsection{Chorus} \songsnippet{Lorem ipsum dolor sit amet, consetetur sadipscing elitr, ...} \optionalChord{(x2)}
	\end{highlightbar}
	
	\songsection{Bridge}
	[A]At vero eos et accusam et justo duo [C]dolores et ea rebum. 
	\vspace{-2.875em} \\ \hfill \strummingTwoBeatsAndFourMovementsPerBeat{dddddddd} \vspace{-0.125em}
	[A]Stet clita [D]kasd gubergren, 
	[A]No sea takimata [C]sanctus est 
	[A]Lorem ipsum dolor[D] sit amet.
	
	% multiple lines instrumental: start in a new line
	\songsection{Instrumental}
	{\footnotesize\begin{tabular}{|ll|ll|ll|ll|}
			Em & D & C & ~ & Em & D & A & ~ \\
			Em & D & C & ~ & Em & D & A & ~ \\
			Em & D & C & ~ & Em & D & A & ~ \\
			Em & D & C & ~ & G & D & C & ~
	\end{tabular}} \optionalChord{(x2)}
	
	\begin{highlightbar}
		\songsection{Chorus} \songsnippet{Lorem ipsum dolor sit amet, consetetur sadipscing elitr, ...}
		\songsnippet{... Labore et dolore magna aliquyam erat,}
		[A]Sed di[C]am [G]vo[A]lup[G]tua.
	\end{highlightbar}
	
	\pagebreak
\comment{about \textbf{comments} in \textbf{Song Section Headers}}
\comment{\textit{grayed out} and in \textit{default font size} for comments and repetition info}
	\songsection{Chorus} \optionalChord{(x2)}
	Lorem ipsum dolor sit amet, consetetur sadipscing elitr, 
	Sed diam nonumy eirmod tempor invidunt ut
	
	\songsection{Verse 1} \optionalChord{(play with muted strings)}
	Lorem ipsum dolor sit amet, consetetur sadipscing elitr, 
	Sed diam nonumy eirmod tempor invidunt ut
	
\comment{"\textit{(N.C.)}" uses a smaller font (footnotesize), same as other chords above the lyrics}
	\songsection{Verse 2} \optionalChord{\footnotesize(\textit{N.C.})}
	Lorem ipsum dolor sit amet, consetetur sadipscing elitr, 
	Sed diam nonumy eirmod tempor invidunt ut
	
	
\end{guitar}