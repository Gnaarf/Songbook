\songtitle{Er wollte wie er sollte}{Versengold}{2005}

\guitarchord{D}
\guitarchord{A}
\guitarchord{Em}
\guitarchord{G}

\begin{guitar}
	\songsection{Strophe 1}
	Er [D]war ein Königs[A]sohn, nicht nur [Em]das, noch er [A]einzig gar.
	Sollt' [D]herrschen über [A]Land, das ihm [Em]stets seine [A]Heimat war.
	[D]Denn sein Vater [A]krank und dem [Em]kommenden [A]Tode nah,
	Sollt' [D]er denn Thrones [A]Erbe nun [G]sein.
	
	Das war ihm nicht genehm und erschien ihm so sonderbar,
	War'n ihm doch Land und Leute seit jeher zu Freunden dar,
	Ritt er doch allzuoft mit der hiesigen Bürgerschar,
	Verliebte sich in Bauers Mägdlein.
	
	\songsection{Refrain}
	Er [D]wollte nicht so [A]sein, wie er [D]sollt', denn er [A]konnte nicht.
	Er [D]konnte nicht so [A]sein, wie er [G]wollt'. O[A]oh-ooh.
	Er [D]wollte nicht so [A]sein, wie er [D]sollt', denn er [A]konnte nicht.
	Er [D]konnte nicht so [A]sein, wie er [G]wollt'.
	
	\songsection{Strophe 2}
	Fühlt' er sich doch als Teil seines Volkes und Landes gar.
	Könnt' er doch niemals knechten, was einst seiner Freundschaft war.
	Denn er guten Herzens und all jenen Leuten nah
	Und wollte ihnen gleichgestellt sein.
	
	Empfand er doch die Steuer und Armut so sonderbar.
	Fand er doch Gold ist flüchtig und wahrlich für alle da.
	Wollt' er doch keine Schuld an der hungernden Bürgerschar.
	Wollt' er doch nur des Bauers Mägdlein.
	
	\songsection{Refrain} \songsnippet{Er wollte nicht so sein...} (x2)
	
	\pagebreak
	
	\songsection{Strophe 3}
	Da kam ihm ein Gedanke der Hoffnung und Einsicht gar,
	Denn, wenn sein Vater tot, er ja Herrscher der Lande war.
	So wollte er besorgen, dass Steuern dem Ende nah
	Und jeder Mann der Freiheit soll sein.
	
	Er wollte niederbrennen, was jeher ihm sonderbar.
	Die Pranger und die Galgen, die Furcht sollte nie mehr dar.
	Er wollte eine glücklich und freudige Bürgerschar;
	Vor allen Dingen Bauers Mägdlein.
	
	\songsection{Refrain}
	Er wollte doch so sein, wie er sollt', denn er konnte es.
	Er konnte doch so sein, wie er wollt'. Ooh-ooh.
	Er wollte doch so sein, wie er sollt', denn er konnte es.
	Er konnte doch so sein, wie er wollt'.	
	
	\songsection{Refrain} \songsnippet{Er wollte doch so sein...}
	
	\songsection{Strophe 4}
	Doch als er sich den Thron nahm, die Krone des Königs gar,
	Da traf er holdes Weib, was ihm jeher versprochen war.
	Sie war so wunderschön und dem Traum seiner Jugend nah,
	Da wollt' er nicht mehr ohne sie sein.
	
	Da schien ihm all sein Denken und Willen so sonderbar;
	Wollt' er sie doch beglücken auf ewig und immerdar.
	So schenkt er ihr Geschmeide auf Kosten der Bürgerschar
	Und vergaß des Bauers Mägdlein.
	
	\songsection{Refrain} \songsnippet{Er wollte nicht so sein...} (x2)
	
\end{guitar}

