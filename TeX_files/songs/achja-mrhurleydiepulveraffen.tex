\songtitle{Ach ja?!}{Mr. Hurley \& Die Pulveraffen}{2015}

\guitarchord{G}
\guitarchord{D}
\guitarchord{C}
\guitarchord{Em}
\guitarchord{B7}

\begin{guitar}
	\songsection{Intro}
	\newcommand{\up}{$\uparrow$}%
	\newcommand{\down}{$\downarrow$}%
	{\footnotesize \setlength{\tabcolsep}{3pt} \begin{tabular}{|cccc|cccc|cccc|cccc|}
			G & ~ & G & D & ~ & ~ & D & ~ & C & ~ & C & D & D & ~ & ~ & ~ \\
			\down & ~ & \down & \up & ~ & ~ & \down & ~ & \down & ~ & \down & \up & \down & ~ & ~ & ~ \\
			1 & ~ & \& & ~ & 2 & ~ & \& & ~ & 3 & ~ & \& & ~ & 4 & ~ & \& & ~ 
	\end{tabular}} (x2)
	
	\songsection{Strophe 1}
	Neulich [G]früh wollt' ich grad [D]aus der Schänke [G]geh'n,
	Da ent[C]stand für mich nur [D]leider ein Pro[G]blem,
	Denn die [G]Wirtin, diese [D]fette, fiese [G]Vettel
	Meint' ich [C]hätt' noch ein paar [D]Rum auf meinem [G]Zettel.
	Ich zog mich [Em]noch am Tresen [D]hoch, dann lallte [G]ich:
	"Ich kann noch [C]steh'n für halbe [D]Sachen zahl' ich [G]nicht."
	Dann machte [Em]mich die saure [D]Seekuh kurzer[G]hand
	Mit ihrem [C]raubeinigen [D]Rausschmeißer be[G]kannt.

	Der sagte: "[B7]Freundchen, dat is' nich' sehr schlau gewesen!
	Gleich [C]kannste dein Gebiss hier von den Dielen [D]lesen!"\vspace{-.1em}
	
	\songsection{Refrain}
	Ach [G]ja!? (Komma [D]her!) Komma [C]näher und ich [D]schwör
	Ich zer[G]stör dein Exter[Em]jöhr und willst du [C]mehr, empfehl ich [D]sehr, dann
	Komma [G]her! (Ach [D]ja!?) Und wir [C]machen das mal [D]klar
	Ja, du [G]laberst von Ge[Em]fahr für mich, doch [C]gar nicht bange [D]sage ich:
	"Na [C]klar, [Em]du und welche [C]Ar[D]ma[G]da?"\vspace{-.1em}
	
	\songsection{Strophe 2}
	Neulich wacht' ich auf und war noch gut dabei
	Auf einem unbekannten Kahn mit Kurs Shanghai.
	Und über mir stand dick und dicht behaart
	Ein mir fremder und potthässlicher Pirat.
	Der meinte "Haste jetzte mal jenuch jeratzt?
	Nu kiek ma' zu, dat du die Planken sauber kratzt!"
	Und ich sagte: "Hör mal zu und mit Verlaub,
	Was ist hier eigentlich los, wer bist du überhaupt?"
	
	"Halt mal den [B7]Rand und mach mal lieber keenen Muckser,
	Wat [C]meenste, wer ick bin? Der Von-der-[D]Planke-Schubser!"
	
	\songsection{Refrain}
	Ach [G]ja!? (Komma [D]her!) Komma [C]näher und ich [D]schwör
	Ich zer[G]stör dein Exter[Em]jöhr und willst du [C]mehr, empfehl ich [D]sehr, dann
	Komma [G]her! (Ach [D]ja!?) Und wir [C]machen das mal [D]klar
	Ja, du [G]laberst von Ge[Em]fahr für mich, doch [C]gar nicht bange [D]sage ich:
	"Na [C]klar, [Em]du und welche [C]Ar[D]ma[G]da?"
		
	\songsection{Pöbelsolo (Melodie aus dem Intro)}
	{\footnotesize \setlength{\tabcolsep}{3pt} \begin{tabular}{|cccc|cccc|cccc|cccc|}
			G & ~ & G & D & ~ & ~ & D & ~ & C & ~ & C & D & D & ~ & ~ & ~ \\
			\down & ~ & \down & \up & ~ & ~ & \down & ~ & \down & ~ & \down & \up & \down & ~ & ~ & ~ \\
			1 & ~ & \& & ~ & 2 & ~ & \& & ~ & 3 & ~ & \& & ~ & 4 & ~ & \& & ~ 
	\end{tabular}}
	
	So, trab mal an, du Seesack! Ich will jetzt mal eins klarkriegen hier! 
	Noch ein mal! Ich sag', noch ein mal und ich pflück dir die Bananen 
	einzeln aus der Staude! Ich takel deine Visage im Sitzen ab, du 
	Küstenschipper! Wenn ich dein Exterjöhr hier noch einen Schlag 
	länger tolerieren muss, dann heißt es aber Feuer frei fürs Faustgewehr! 
	Du kannst hier gleich mit gebrochenen Fingern deine Zähne aus den 
	Planken puhlen, weißt du das? Ich sag' ich zähl bis eins und dann
	ist dein Achterdeck hier ablandig, du Landratte!
	
	\songsection{Strophe 3}
	Der [G]aufmerksame [D]Hörer fragt sich [G]schon:
	"Was [C]soll denn diese [D]tumbe Aggres[G]sion?
	Diese Pi[G]raten haben ja wohl [D]{ü}berhaupt kein [G]Hirn."
	Doch auf [C]sowas können wir nur respon[D]dier'n:
	
	\songsection{Refrain} \songsnippet{Ach ja?! ...} (x2)
	
\end{guitar}