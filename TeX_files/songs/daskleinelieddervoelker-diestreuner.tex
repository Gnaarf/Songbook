\songtitle{Das kleine Lied der Völker}{Die Streuner}{2000}

\guitarchord{C}
\guitarchord{G}
\guitarchord{Am}
\guitarchord{Em}
\guitarchord{F}

\begin{guitar}
	\songsection{Strophe 1}
	Der [C]Schwachsinn ruft, der [G]Gaukler naht, [Am]denn er ist ganz [Em]schwer auf Draht
	Bei [F]{ü}blen Witzen und [C]Hochverrat [G]springt der König im Quadrat.
	
	\begin{highlightbar}
		\songsection{Refrain}
		[C]Jab-dab-da, Da-ba-[F]da-ba-dab-dai. Ja-ba-[C]dab-dab, Da-ba-da, [G]Da-ba-dab-dai.
		[C]Jab-dab-da, Da-ba-[F]da-ba-dab-dai. Ja-ba-[C]dab-dab, [G]Da-ba-dab, [C]Dai.
	\end{highlightbar}
	
	\songsection{Strophe 2}
	Der König hat die Krone auf mit bunten Steinen oben drauf
	Und treibt er's mal gar zu munter, fällt die Krone eben runter.
	
	\begin{highlightbar}
		\songsection{Refrain} \songsnippet{Jab-dab-da Da-ba-da-ba-dab-dai. ...}
	\end{highlightbar}
	
	\songsection{Strophe 3}
	Die Hütte brennt, die Fee ist drin. Ich rette sie, na immerhin.
	Der Streuner liebt das Risiko und brennt er jetzt auch lichterloh.
	
	\begin{highlightbar}
		\songsection{Refrain} \songsnippet{Jab-dab-da Da-ba-da-ba-dab-dai. ...}
	\end{highlightbar}
	
	\songsection{Strophe 4}
	Ein Vampir als Fledermaus dachte sich: "Flieg gradeaus"
	Er sah den Baum doch nicht das Tor. Jetzt singt er im Knabenchor.
	
	\begin{highlightbar}
		\songsection{Refrain} \songsnippet{Jab-dab-da Da-ba-da-ba-dab-dai. ...}
	\end{highlightbar}
	
	\songsection{Strophe 5}
	Der Meuchler macht die Leute kalt für Geld und aus dem Hinterhalt.
	Muss er sich ins Grab nun legen - auch ein Meuchler hat Kollegen.
	
	\begin{highlightbar}
		\songsection{Refrain} \songsnippet{Jab-dab-da Da-ba-da-ba-dab-dai. ...}
	\end{highlightbar}
	
	\songsection{Strophe 6}
	Der Graf, das Schaf, war immer brav, singt die Kinder in den Schlaf.
	Die Gräfin nachts ist nicht zu seh'n. Tja, ihr Leute, so kann's geh'n.
	
	\begin{highlightbar}
		\songsection{Refrain} \songsnippet{Jab-dab-da Da-ba-da-ba-dab-dai. ...}
	\end{highlightbar}
	
	\songsection{Strophe 7}
	Der Waldläufer im grünen Rock nimmt für'n Bogen einen Stock.
	Doch der lässt sich nicht lange biegen. Nun sieht man die Zähne fliegen.
	
	\begin{highlightbar}
		\songsection{Refrain} \songsnippet{Jab-dab-da Da-ba-da-ba-dab-dai. ...}
	\end{highlightbar}
	
	\songsection{Strophe 8}
	Die Hexen reiten auf dem Besen. Ja, so ist es stets gewesen.
	Doch hab ich mal nachgedacht. Wo steckt der Besen in der Nacht?
	
	\begin{highlightbar}
		\songsection{Refrain} \songsnippet{Jab-dab-da Da-ba-da-ba-dab-dai. ...} \optionalChord{(x2)}
	\end{highlightbar}
	
\end{guitar}