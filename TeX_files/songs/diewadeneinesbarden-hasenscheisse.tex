\songtitle{Die Waden eines Barden}{Hasenscheisse}{2007}

\guitarchord{Am}
\guitarchord{C}
\guitarchord{E}
\guitarchord{Dm}
\guitarchord{F}

\begin{guitar}
	
	\songsection{Strophe 1}
	[Am]Neulich saß ich in der Schänke und genoss dort die Getränke,
	Als der [C]Schatten eines großen, fiesen Mannes auf mich [E]fiel! (Whoo$\sim$)
	[Am]Ungebeten nahm er Platz, fasste mich an meinem Latz
	Und er[C]zählte mir von Ehre, Kameradschaft, Diszi[E]plin. (Whoo$\sim$)
	[Am]Und da war mir [C]plötzlich klar, wer [E]dieser düst're [Am_]{Un}[E]hold [Am]war.
	[Am]Und da war mir [C]plötzlich klar, wer [E]dieser düst're [Am_]{Un}[E]hold [Am]war.
	
	\songsection{Refrain}
	Er [Am]wollte mich zum Kriege schicken, doch ich sagte zu ihm: "[E]Ficken!"
	Denn das sind die [Am]Waden eines [E]Barden 
	Und die sind nicht zum Mar[Am]schier'n.
	Denn wenn sich [Am]alle Barden [E]schlagen, wer soll da noch musi[Am]zier'n?
	Und ganz [Dm]ohne die Musike, was gibt's [Am]da noch zu verlier'n?
	Das sind die [E]Waden eines Barden und die sind nicht zum Mar[Am]schi-[E]i-[Am]ier'n,
	Zum Mar[Am]schi-[E]i-[Am]ier'n
	
	\songsection{Strophe 2}
	Doch der Unhold gab nicht auf und griff nach seiner Klinge Knauf.
	Und setzt mir an den Hals sein scharfes Schwert. (Whoo$\sim$)
	Er schaute fest in meine Augen: "Habt Ihr denn gar keinen Glauben?
	Fühlt Ihr Euch als Soldat denn nicht geehrt? (Whoo$\sim$)
	Was seid Ihr denn für ein Mann, der nicht mal richtig töten kann?"
	Ich stieß ihn fort, ich hatte Durst. "Hört, Euer Krieg, der ist mir Wurst!"
	
	\songsection{Refrain} \songsnippet{Er wollte mich zum Kriege schicken...}
	
	\songsection{Strophe 3}
	Nun war's mit dem Spaß vorbei, nenn' wir es mal Phase drei,
	Der Mann erlitt einen Tobsuchtsanfall! (Whoo$\sim$)
	Er zerkaute einen Hocker und verschlang sein Schwert ganz locker
	Und kackte in die Ecken überall. (Whoo$\sim$)
	Und da war mir plötzlich klar, dass dieser Mann bescheuert war!
	Und da war mir plötzlich klar, dass dieser Mann bescheuert war!
	
	\songsection{Refrain} \songsnippet{Er wollte mich zum Kriege schicken...}
	
	\songsection{Bridge}
	Für kein [Dm]Geld dieser [Am]Welt schwinge [E]ich für euch die [Am]Lanze,
	Sondern [Dm]höchstens meine [Am]Waden, aber [E]die auch nur zum [Am]Tanze.
	Für kein' [F]Ruhm, für keine Ehre, für [C]keinen Hungerlohn
	Und schon [F]gar nicht für den König, denn der [Am]bangt um seinen Thron.
	Für kein' [F]Ruhm, für keine Ehre, für [C]keinen Hungerlohn
	Und schon [F]gar nicht für den König, denn der [Am]bangt um seinen Thron.
	Nein, ich [C]lasse mich nie[Dm]mals von euch in keine Kriege [Am]schicken.
	Da [C]sage ich: "Nein, [Dm]danke!" und zum Abschied nochmal "[E]Ficken!"
	
	\songsection{Refrain}
\end{guitar}

\begin{music}
	\instrumentnumber{1}
	\generalsignature{-1}
	\setlines16\setclefsymbol1\tabclef\setsize1\largevalue\setsign10
	\nobarnumbers
	\parindent=2ex
	\startextract
	\Notes%
		\ltab1{7}%
		\ltab1{8}%
		\ltab1{10}%
		\ltab1{8}%
		\ltab1{7}%
		\ltab1{5}%
		\ltab1{7}%
		\ltab1{5}%
		\ltab1{3}%
		\ltab1{2}\en
	\NOTes%
		\zcn{-5}{Denn}\ltab1{0}%
		\zcn{-5}{das}\ltab2{4}%
		\zcn{-5}{sind}\ltab2{1}%
		\zcn{-5}{die...}\ltab3{2}\en
	\zendextract
\end{music}

\begin{guitar}
	\linebreak
	... [Am]Waden eines [E]Barden und die sind nicht zum Mar[Am]schier'n,
	Denn wenn sich [Am]alle Barden [E]schlagen, wer soll da noch musi[Am]zier'n?
	Und ganz [Dm]ohne die Musike, was gibt's [Am]da noch zu verlier'n?
	Drum küsst die [E]Waden eurer Barden, vergesst nie sie zu mas[Am]sier'n.
	Küsst ihren [Dm]Mund, küsst ihren [Am]Arsch und küsst [E]ihre süßen [Am]Schenkel.
	Küsst den [Dm]Opi und die [Am]Omi und ver[E]gesst nicht ihre [Am]Enkel.
	Hegt und [F]pflegt sie, eure Barden, und er[Am]füllt ihn'n jeden Wunsch.
	Reicht ihnen [E]Bier und reicht ihnen Met,
	Aber niemals Früchte[Am]punsch! [E ] [Am]{ } 
\end{guitar}


