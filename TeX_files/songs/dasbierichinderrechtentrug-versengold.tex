\songtitle{Das Bier ich in der Rechten trug}{Versengold}{2007}

\guitarchord{D}
\guitarchord{A}
\guitarchord{G}

\begin{guitar}
	\songsection{Strophe 1}
	Ich [D]schlenderte ge[A]mach, versonnen [G]aus der Schänkentür.
	Mit [D]Armen voller [A]Freudenwonnen [G]lag die Nacht vor mir.
	In [D]meinem Mund ein [A]Pfeifchen hin, im [G]linken Arm ein Mägdlein ging,
	In [D]rechter Hand ein [A]Krug voll Bier, so [G]wandelten hin[D]aus wir vier.
	
	Doch als ich auf die Straße trat, voll Frohgemut und -sinn,
	Schritt ich in schlammig' Stadtunrat und schlitterte dahin.
	Der Untergrund geschwind entglitt, im Schwung nahm ich das Mägdlein mit,
	Die mir im Schreck und ihrem Flug das Pfeifchen aus dem Munde schlug.
	
	\songsection{Refrain}
	Die [G]Welt, sie hielt den [A]Atem an, die [D]Zeit stand stockend [D]still
	Und [G]ich ersann, was [D]ich noch retten [A]kann und retten [D]will.
	
	\songsection{Strophe 2}
	Ich stützte mich mit linker Hand und warf mich hoch empor
	Und zog dabei nicht grad galant am Haar das Mägdlein vor.
	Dann trat ich mit dem Fuß die Pfeif', die flog in einem Funkenreif
	Hinweg der Magd, die grad nach vorn, wie ich erneut den Halt verlor'n.
	
	Ich warf mich also auf den Rücken, und mit linker Hand und Knie
	Tat ich sie wuchtig von mir drücken, daß sie rittlings fiel und schrie.
	Grad noch erreichte denn mein Schuh das Pfeifchen, und ich trat schnell zu,
	So sauste sie erneut hinweg dem Weib, sich nähernd Straßendreck.
	
	\songsection{Refrain} \songsnippet{Die Welt, sie hielt den Atem an, die Zeit stand stockend still (...)}
	
	\songsection{Strophe 3}
	Ich schwang mein' Oberleib hinauf und hielt und riss die Magd am Kleid.
	Das hat sie zwar nicht von dem Sturz, doch von dem schnöden Kleid befreit.
	Dann wollt' ich, daß mein Munde fing das Pfeifchen, das zu Boden ging.
	So beugte ich mein Kreuze krumm und fing es zwar, doch falsch herum.
	
	Voll Schmerz gepeint spie ich die Glut im allzu weiten Bogen aus
	Und streckte mich voll Übermut mit letzter Kraft in Saus und Braus
	In Richtung Magd, die leuchtend gar mit meiner Funkenglut im Haar,
	Trotz all der Müh', die ich mir gab, fiel klatschend in den Stadtunrat.
	
	\songsection{Refrain} \songsnippet{Die Welt, sie hielt den Atem an, die Zeit stand stockend still (...)}
	
	\songsection{Strophe 4}
	So stand ich denn betreten da, von Schlamm und Matsch benetzt,
	Besudelt, stinkend, muffig gar, vom Straßendreck durchsetzt,
	Vor einer Magd, die halbnackt war, mich schmorend und verletzt besah
	Und trotzig sich denn abgewandt, ist schluchzend sie nach Haus gerannt.
	
	Auch mein guter Tabak war in aller Welt verstreut.
	Mein guter, edler Tabak, den genießen wollt' ich heut.
	Dahin war die erhoffte Nacht, so hab ich mich denn heimgemacht
	Und trank frustriert in einem Zug das Bier ich in der Rechten trug.
\end{guitar}