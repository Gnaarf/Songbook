\songtitle{Cinderella}{EAV}{1994}

\guitarchord{D}
\guitarchord{A}
\guitarchord{G}
\guitarchord{Bm}
\guitarchord{Bm/A}
\guitarchord{Bm/G}
\guitarchord{F}
\guitarchord{C}
\guitarchord{Dsharp}
\guitarchord{Asharp-2}
\guitarchord{Gsharp}
\guitarchord{Cm}
\guitarchord{Cm/Asharp}
\guitarchord{Cm/Gsharp}
\guitarchord{Fsharp}
\guitarchord{Csharp-2}

\begin{guitar}
	\songsection{Strophe 1}
	Es [D]lebte einst ein schönes Mädel. Cinderella [A]war ihr Nam'.
	[D]Und es warte[G]te vergebens [A]auf den Prinz der [D]niemals kam.
	
	[D] Sie schlief im Kohlenkeller. Trotzdem war sie [A]bettelarm,
	[D]Weil sie von der [G]vielen Kohle, [A]die da lag, zu[D]wenig nahm
	
	[Bm]Aschenbrödel, [Bm/A]sei kein Blödel! V'[Bm/G]trödel [A]keine [Bm]Zeit!\vspace{-0.3em}
	
	\songsection{Instrumental} {\footnotesize\begin{tabular}{l|l|l|l}
			D & D & D & A \\
			D & G & A & D 
	\end{tabular}}
	\songsection{Strophe 2}
	Eines Abends, so um sieben, schlich sie heimlich sich hinfort,
	Um ein Tänzchen kurz zu schieben - im Nachbarsort
	
	Steinig wars, guru guru. Sie hatte keinen Schuh
	
	Auf dem Schlosse angekommen - Gold, Geschmeide, Sakrament -
	Und sie hörte ganz benommen die Gebrüder Grimmig Band. 
	\hfill- Ans zwa drei vier\vspace{-0.5em}
	\songsection{Refrain}
	[D]Komm, Cinder[G]ella, [D]hol die Wurst vom [A]Keller. 
	[D]Komm, Cinder[G]ella, be[Bm]vor die Turmuhr [A]läut'.
	[D]Komm, Cinder[G]ella, [D]mach ein wenig [A]schneller, 
	[F]Denn bis zur [C]Mitte der [G]Nacht ist nur mehr wenig [A]Zeit.\vspace{-0.3em}
	
	\songsection{Strophe 3}
	An der Bar steht Prinz von Ölen. Er ist wieder grausam zu
	Und alsbald hört man ihn grölen: "Schönes Kind wer bis denn du?"
	
	"Bin ein armes Findelkindel. Habe keine Schuh."
	
	"Was bis du? Ein Findelkindel? Bargeldlos und ohne Schuh?",
	Rief der Prinz, "Das klingt nach G'sindel. Freunde was sagt ihr dazu?"
	
	Armes Kind hier hast du was. Kauf die zwei neue Adidas.
	
	\songsection{Strophe 4}
	Prinz von Ölen zeigt Erregung, stiert sie roten Auges an,
	Erwägt die Temporärbelegung, formuliert den Beischlafplan
	
	Sagt zu ihr: "Guru guru, Mädel, hör gut zu."
	
	"Du marschierst jetzt in den Keller und dort fährst du aus dem Kleid.
	Mach die Fliege, den Propeller. Ich trink noch eine Kleinigkeit."
	\hfill- Prost!\vspace{-0.3em}
	\songsection{Refrain} \songsnippet{Komm, Cinderella, hol die Wurst vom Keller.}
	
	\songsection{Instrumental} {\footnotesize\begin{tabular}{l|l|l|l}
			D$\sharp$ & D$\sharp$ & D$\sharp$ & A$\sharp$ \\
			D$\sharp$ & G$\sharp$ & A$\sharp$ & D$\sharp$ 
	\end{tabular}}
	
	\songsection{Strophe 5}
	[D#]Cindy wartet drunt im Dunkeln, steigt aus ihrem [A#]Jutestoff.
	[D#]Oben ist der [G#]Prinz am Schunkeln, [A#]weil er den Ter[D#]min versoff.
	
	[D#]Also eilt sie flink nach oben nach zwei Stunden [A#]voller Zorn.
	[D#]Einen Zacken [G#]in der Krone [A#]hat der Prinz und [D#]fällt nach vorn.
	
	[Cm]Aschenbrödel, [Cm/A#]sei kein Blödel. [Cm/G]Zock ihn [A#]ab, den [Cm]dumpfen Dödel!
	
	\songsection{Strophe 6}
	Cinderella sprach: "Herr Ölen, Hoheit, sind erstaunlich fett.
	Demnächst wernd wir uns vermählen und dann ab ins Himmelbett!"
	
	"Was [Cm]sagst du da? Laut [Cm/A#]Märchen [Cm/G#]werden [A#]wir ein [Cm]Pärchen?"
	[G#]Märchen gehen - [D#]eiderdaus - in [Cm/G#]Wahrheit [A#]anders [G]aus.
	
	Es sprach der Prinz zu Cinderella: "Marsch nach Hause! Hopp und zack!
	Ab in deinen Kohlenkeller! Morgen ist ein harter Tag!"
	\hfill- Ans zwa drei vier\vspace{-0.3em}
	\songsection{Refrain}
	[D#]Komm, Cinder[G#]ella, springt [D#]auch die Wurst vom [A#]Teller.
	[D#]Komm, Cinder[G#]ella, [Cm]nimm es nicht so [A#]schwer.
	[D#]Komm, Cinder[G#]ella, [D#]weine nicht im [A#]Keller.
	[F#]In Hundert [C#]Jahren da [G#]gibt's dich sowieso nicht [A#]mehr.
	
	\songsection{Outro} {\footnotesize\begin{tabular}{l|l|l|l}
			D$\sharp$ & D$\sharp$ & D$\sharp$ & A$\sharp$ \\
			D$\sharp$ & G$\sharp$ & A$\sharp$ & D$\sharp$ 
	\end{tabular}}
\end{guitar}