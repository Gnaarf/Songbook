\songtitle{Faul \& Fett}{Hasenscheisse}{2007}
% source: based on chordify

\guitarchord{Bm}
\guitarchord{A}
\guitarchord{D}
\guitarchord{G}
\guitarchord{Fsharpm}
\guitarchord{C}

\begin{guitar}
	\songsection{Refrain}
	Wir liegen [Bm]faul und [A]fett im [D]Gras
	Wir saufen [Bm]Bier, das [A]macht uns [D]Spaß
	
	\songsection{Strophe 1}
	Und [Bm]unter uns im [D]tiefen Erdreich, [G]dort verfault der [F#m]Feind
	Er [Bm]liegt dort kalt und [D]feucht, wir liegen [G]wo die Sonne [F#m]scheint
	Er [G]dient uns dort als [D]Dung fürs weiche [Bm]Gras, auf dem wir liegen
	Auf [G]dem wir süße Tr[D]{ä}ume träum'n und [Bm]Kinder friedlich wiegen
	
	\songsection{Refrain} \songsnippet{Wir liegen faul und fett im Gras...}
	
	\songsection{Strophe 2}
	Und neben uns beim Nachbarn ist das Gras noch ziemlich kurz
	Wir rufen zu ihm rüber: "Alter sag mal, was denn los?"
	Und er erzählt uns dass der Knabe, der da unter ihm begraben
	Noch nicht so lange tot ist und wir soll'n nicht so viel fragen
	
	Wir fragen aber trotzdem heiter weiter: "Hey jetzt sei mal nicht so faul!
	Du erzählst uns die Geschichte oder 's gibt nen paar aufs Maul!"
	Na bei soviel Überredungskunst, da ließ er sich nicht lumpen
	Und er kam zu uns rüber mit zwei, drei gefüllten Humpen
	
	\songsection{Refrain} \songsnippet{Wir Lumpen liegen faul und fett im Gras...}
	
	\songsection{Strophe 3}
	Und er zieht uns sauber all die Humpen übern Schädel
	"Das ist für eure Neugier, ihr vermaledeiten Flegel!"
	Dann holte dieses Tier noch einen Spaten aus dem Haus
	Und sagt er muss uns jetzt begraben, weil wir seh'n so scheiße aus
	Wir seh'n so scheiße …
	
	Momentchen also jetzt mal Stop, det kann ja wohl nich sein
	Wir seh'n ja wohl nich scheiße aus, wer det sagt is jemein
	Nun schaut euch mal den Matze an, ein Mann wie Wilhelm Tell
	Für Heldenstatuen stand der Junge öfter schon Modell
	%Und er, er selbst, Chrimas der Schelm, hat zwar ne kleene Meise
	%Er bricht dafür die Frauenherzen aber reihenweise
	Und [Bm]er, er selbst, Chri[D]mas der Schelm, hat [G]zwar ne kleene [F#m]Meise
	Er [Bm]bricht dafür die [D]Frauenherzen [G]aber reihen[F#m]weise
	
	Aber jut jetzt Schluss damit, zurück zum Wesentlichen
	Es sah nicht rosig aus für uns, die Lage war beschissen
	Unser Nachbar, seelenruhig, grub seine Löcher tiefer
	Und wir lagen bewusstlos da, mit halbjebroch'nem Kiefer lagen wir...
	
	\songsection{Refrain}
	… faul und fett im Gras
	Mit Blut beschmiert und voll im Arsch
	
	\songsection{Strophe 4}
	Menschenskinder, wir befanden uns in allergrößter Not
	Jetzt musste was passier'n ganz schnell, sonst sind wir alle tot
	Mit janz viel Überzeugungskraft und janz viel Energie
	Ham wat schließlich och jeschafft, na fragt uns bloß nich wie
	
	Man kann dazu nur soviel sagen ... nee man kanns jar nich erklär'n
	Es war so unbegreiflich – wir erzähln's immer wieder gern
	Wir rüttelten uns quasi aus der Auennacht wieder wach
	Und nahmen uns den Kerl zur Brust, doch der hielt uns in Schach
	
	Der Typ hatte sich nämlich einen Keller eingerichtet
	Sich mit Trainieren und Hanteln stemmen, die Arme neu beschichtet
	Der Kampf dauerte, was glaubst du da, so an die drei, vier Stunden
	Er hatte seine Höh'n und Tiefen und ging über zwölf Runden
	
	In Runde zwölf, da wurde mir ganz allerplötzlichst klar
	Dass ich ein Säckchen Zauberpulver mit mir führte – ha!
	"Hasenscheisse" "Zauberpulver"
	Also hurtig raus damit dem Schurken ins Jesichte
	Wat soll schon sein, er starb daran – so endet die Jeschichte
	
	Nein jetzt im Ernst, er wurde grün, ganz klein und immer dünner
	Sein schurkenmäßijet Jebrüll war plötzlich nur Jewimmer
	Verwandelt in ein Büschel Gras – erbärmlichst anzuschau'n
	Na mit sowat mussten wir uns ja nun wirklich nich mehr hau'n
	
	\songsection{Outro}
	Und [Bm]einmal, ja, da uri[D]nierte ich hi[G]nein mit größter [F#m]Wonne
	[Bm]War ja später [D]trocken wieder, al[G]lein schon durch die [C]Sonne[Bm]{}
	Und die [Bm]Moral von der Ge[D]schichte: wir [G]hatten großen [F#m]Spaß
	Schön [Bm]dass ihr alle [D]da wart, ge[G]segnet sei das [Bm]Gras
\end{guitar}