\songtitle{Wein, Weib und Gesang}{Die Streuner}{1998}

\guitarchord{G}
\guitarchord{D}
\guitarchord{Am}
\guitarchord{Em}
\guitarchord{C}
\guitarchord{D7}

\begin{guitar}
	\songsection{Refrain}
	[G]Wein, [D]Weib und [G]Gesang
	Und das [G]Ganze ein Leben [D]lang.
	Wenn [Am]das nicht mehr wär, ich [Em]armer Tor,
	Dann [C]wär mir [D7]Angst und [G]Bang.
	Ja, dann [C]wär mir [D7]Angst und [G]Bang.
	
	\songsection{Strophe 1}
	Schlaget [G]an das [D]erste [G]Fass,
	Dann der [G]Wein schlichtet größten [D]Hass.
	Er be[D]nebelt die Sinne und schlägt auf die Stimme;
	Aus [C]jedem Te[D7]nor wird ein [G]Bass.[D]{}
	
	\songsection{Refrain} \songsnippet{Wein, Weib und Gesang...}
	\songsection{Strophe 2}
	Mannen hebet an den Kilt.
	Für die Weiber ein lustiges Bild.
	Doch wer sich nicht traut, weil er klein ist, lieber schaut,
	Verstecke sich hinter sein Schild.
	
	\songsection{Refrain} \songsnippet{Wein, Weib und Gesang...}
	\songsection{Strophe 3}
	Weiber knöpft auf euer Hemd aber schnell,
	Denn wir Mannen lieben Blusen ohne "l".
	Bleibt das Hemd zu bis oben, kriegt ihr kein' Mann zum Toben.
	Tut ihr's doch gibt's Gejaul und Gebell.
	
	\songsection{Refrain} \songsnippet{Wein, Weib und Gesang...}
	\songsection{Strophe 4}
	Ja, das Lied hat mir Spaß gemacht,
	Doch ich seh' es hat nichts gebracht,
	Drum pack ich die Laute und spiel' ander'n Leuten
	Meine ganze Liederpracht.
	
	\songsection{Refrain} \songsnippet{Wein, Weib und Gesang...}
\end{guitar}